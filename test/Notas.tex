parser.csv es la data cruda.

A eso se le aplica el KGS_filtered.py el cual simplemente agarra la data de interes quedandonos en casos genericos.
Solo se queda con tamaños de tablero comunes.

Al CSV que sale se le aplica KGS_filterd_julia.py el cual este ya trata la base de datos para
los experimentos que se quieren en una primera instancia.

Teniendo la base de datos usamos diferentes archivos de julia para generarnos los diferentes arichivos
CSV de inforamcion de trueskill.

Con estos CSV ya de informacion de trueskill vuelvo a levantar con python para realizar las diferentes figuras.

Julia:
TTT_datos.jl corre simplemente TTT
beta_gamma.jl va a realizar una matriz de log evidencia para diferentes betas y gammas

Python:
PlayerSigma.py genera la evolucion de mu y de sigma para algunos jugadores en funcion del numero de partidas
Dan.py genera la evolcuion de rango oficial.
handicap19.py genera la figura de estimacion  de handicap

